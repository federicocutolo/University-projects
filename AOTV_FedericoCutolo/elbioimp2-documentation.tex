\documentclass[UKenglish]{elbioimp2}
\usepackage[utf8]{inputenc}

\newcommand{\lacmd}[1]{\p{\textbackslash #1}}
\newcommand{\lacmdx}[1]{\lacmd{#1\{\dots\}}}
\newcommand{\p}[1]{\texttt{#1}}

\title{\texttt{elbioimp2} documentation}
\shorttitle{\tt{elbioimp2} doc}
\author{Dag Langmyhr\affiliation{Department of Informatics, University
    of Oslo, Norway \url{dag@ifi.uio.no}}}
\shortauthor{Langmyhr}

%\elbioimpdoi{0}
%\elbioimpreceived{20 Sep 2020}
%\elbioimppublished{21 Sep 2020}
%\elbioimpyear{2020}
%\elbioimpvolume{1}

\usepackage[backend=biber,style=vancouver]
    {biblatex}
\usepackage{csquotes}
\addbibresource{demo.bib}


%% And, now, the article itself:

\begin{document}
\maketitle

\begin{abstract}
  The \p{elbioimp2} document class has been designed for using
  \LaTeX{} to write articles for the \emph{Journal of Electrical
    Impedance}. 

  \keywords{elbioimp2, \LaTeX}
\end{abstract}


\section{Background}
This document class is for writing manuscripts for the \emph{Journal
  of Electrical Impedance} in the second, redesigned version, as
shown in this documentation. (The now obsolete document class
\p{elbioimp} is for the old design.)

\section{How to use the document class}
Since the \p{elbioimp2} document class is based on the standard
\p{article} class, most constructs from \p{article} may be used in
\p{elbioimp2} as well.

\subsection{Additional commands}
In addition to the standard \LaTeX{} commands available in the
\p{article} document class, the following ones are available:
\begin{description}
\item[\lacmdx{affiliation}] is used just after an author's name to
  specify his or her affiliation. (It is used only in the
  \lacmd{author} command.) 

\item[\lacmdx{elbioimpdoi}] provides the document's DOI
  number.

\item[\lacmdx{elbioimpfirstpage}] gives the page number of the first
  page of the article (if other than~1).

\item[\lacmdx{elbioimppublished}] is used for the publishing date of
  the article.

\item[\lacmdx{elbioimpreceived}] shows when the publishers received
  the article.

\item[\lacmd{elbioimpvolume}] gives the volume number in which the
  article is published.
  
\item[\lacmdx{elbioimpyear}] indicated the year of publication (if
  other than the current year).

\item[\lacmdx{keywords}] lists the keywords. (It is only used in the
  \p{abstract} environment.)

\item[\lacmdx{shortauthor}] is for the first author's last name to use
  in the headings..

\item[\lacmdx{shorttitle}] provides a short version of the title, also
  to use in the headings..
\end{description}

\subsection{Reference style}
The reference list will be typeset according the \emph{Vancouver}
style.\cite{biomed-req} To achieve this, you must load Bib\LaTeX{}
this way in your document:

\begin{verbatim}
\usepackage[backend=biber,style=vancouver]
    {biblatex}
\usepackage{csquotes}
\addbibresource{mybib.bib}
\end{verbatim}

\printbibliography
\end{document}
